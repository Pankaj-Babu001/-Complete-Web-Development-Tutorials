\usepackage{hyperref}Lecture 30
\chapter*{Events in Javascript | Even and Event handler in Javascript | Javascript Full Course #15}

\href{https://www.youtube.com/@CoderArmy9}{}

%🎓 WHAT YOU WILL LEARN IN THIS LECTURE:
%Part 1: The Fundamentals of Events
%What is a DOM Event? The "doorbell" analogy.
%The 3 Core Components: Target, Event Type, and Handler.
%The Modern Way: Mastering addEventListener().
%Why the old onclick attribute is not recommended (Separation of Concerns).
%The powerful event object: Your "information packet" for every event.
%Part 2: The Three Phases of an Event
%Event Bubbling Explained: Why events "bubble up" from the child to the parent.
%Event Capturing Explained: The opposite of bubbling, where events "trickle down" from the top.
%A complete, step-by-step trace of the entire event flow.
%How to control the flow with event.stopPropagation().
%Part 3: The Event Delegation Pattern
%What is Event Delegation and why is it so efficient?
%Using event.target vs. event.currentTarget to identify which child was clicked.
%A practical, real-world example of building a dynamic list with a single event listener.
%
%
%<--------------------------------------------------------------------------------------------------------------------------->
%
%🎓 Part 1: The Fundamentals of Events
%What is a DOM Event? The "Doorbell" Analogy
%
%
%┌─────────────────────────────────────────────┐
%│                 OUTSIDE                     │
%│  Someone rings the doorbell (Event occurs)  │
%│                                             │
%│    ┌─────────────────────────────────────┐  │
%│    │               HOUSE                 │  │
%│    │  ┌─────────────┐ ┌───────────────┐  │  │
%│    │  │   TARGET    │ │   HANDLER     │  │  │
%│    │  │  (Button)   │ │ (You answer   │  │  │
%│    │  │             │ │   the door)   │  │  │
%│    │  └─────────────┘ └───────────────┘  │  │
%│    │                                     │  │
%│    │    EVENT TYPE: "doorbell ring"      │  │
%│    └─────────────────────────────────────┘  │
%└─────────────────────────────────────────────┘
%
%
%Components:
%- Target: Which element was interacted with (the doorbell button)
%- Event Type**: What happened (click, keypress, etc.)
%- Handler: Function that runs when event occurs
%
%Modern Way: addEventListener()
%
%javascript
%// OLD WAY (not recommended)
%button.onclick = function() { /* handler */ };
%
%// MODERN WAY
%button.addEventListener('click', function(event) {
%    // event object contains all information
%});
%
%Separation of Concerns**: Keep HTML (structure), CSS (style), and JavaScript (behavior) separate.
%
%The Event Object - Your "Information Packet"
%
%
%┌─────────────────────────────────────────────┐
%│              EVENT OBJECT                   │
%│ ┌─────────────────────────────────────────┐ │
%│ │            event.target                 │ │ ← Element that triggered event
%│ │ ┌─────────────────────────────────────┐ │ │
%│ │ │        event.currentTarget          │ │ │ ← Element that has the handler
%│ │ │ ┌─────────────────────────────────┐ │ │ │
%│ │ │ │       event.type                │ │ │ │ ← Type of event ('click')
%│ │ │ │ ┌─────────────────────────────┐ │ │ │ │
%│ │ │ │ │    event.preventDefault()   │ │ │ │ │ ← Stop default behavior
%│ │ │ │ │ ┌─────────────────────────┐ │ │ │ │ │
%│ │ │ │ │ │ event.stopPropagation() │ │ │ │ │ │ ← Stop event bubbling
%│ │ │ │ │ └─────────────────────────┘ │ │ │ │ │
%│ │ │ │ └─────────────────────────────┘ │ │ │ │
%│ │ │ └─────────────────────────────────┘ │ │ │
%│ │ └─────────────────────────────────────┘ │ │
%│ └─────────────────────────────────────────┘ │
%└─────────────────────────────────────────────┘
%
%🎯 Part 2: The Three Phases of an Event
%
%Event Flow Visualization
%
%
%┌─────────────────────────────────────────────┐
%│                      WINDOW                 │  ← CAPTURING starts here
%│    ┌─────────────────────────────────────┐  │
%│    │                DOCUMENT             │  │
%│    │    ┌─────────────────────────────┐  │  │
%│    │    │             HTML            │  │  │
%│    │    │    ┌─────────────────────┐  │  │  │
%│    │    │    │        BODY         │  │  │  │
%│    │    │    │   ┌─────────────┐   │  │  │  │
%│    │    │    │   │    PARENT   │   │  │  │  │
%│    │    │    │   │  ┌───────┐  │   │  │  │  │
%│    │    │    │   │  │CHILD ◄├──┼───┼──┼──┼──┼── ACTUAL TARGET (click occurs)
%│    │    │    │   │  └───────┘  │   │  │  │  │
%│    │    │    │   └─────────────┘   │  │  │  │
%│    │    │    └─────────────────────┘  │  │  │
%│    │    └─────────────────────────────┘  │  │
%│    └─────────────────────────────────────┘  │
%└─────────────────────────────────────────────┘
%↓ CAPTURING PHASE                        ↑ BUBBLING PHASE
%(trickle down)                           (bubble up)
%
%Step-by-Step Event Flow
%
%1. CAPTURING PHASE** (Window → Target):
%
%Window → Document → HTML → Body → Parent → Child (Target)
%
%2. TARGET PHASE**:
%
%Handler executes on the actual target element
%
%3. BUBBLING PHASE** (Target → Window):
%
%Child (Target) → Parent → Body → HTML → Document → Window
%
%Controlling Event Flow
%
%javascript
%// Stop event from propagating further
%element.addEventListener('click', function(event) {
%    event.stopPropagation(); // Prevents bubbling/capturing
%    // event.stopImmediatePropagation() - stops all handlers
%});
%
%// Use capturing phase (instead of bubbling)
%element.addEventListener('click', handler, true);
%// OR
%element.addEventListener('click', handler, {capture: true});
%
%
%🔄 Part 3: Event Delegation Pattern
%
%Why Event Delegation?
%
%Without Delegation (Inefficient):
%
%┌─────────────────────────────────────────────┐
%│                 PARENT                      │
%│  ┌─────┐ ┌─────┐ ┌─────┐ ┌─────┐ ┌─────┐    │
%│  │  A  │ │  B  │ │  C  │ │  D  │ │  E  │    │
%│  │ (×) │ │ (×) │ │ (×) │ │ (×) │ │ (×) │    │
%│  └─────┘ └─────┘ └─────┘ └─────┘ └─────┘    │
%│  5 separate event listeners = Inefficient!  │
%└─────────────────────────────────────────────┘
%
%
%With Delegation** (Efficient):
%
%┌─────────────────────────────────────────────┐
%│                 PARENT                      │
%│  ┌─────┐ ┌─────┐ ┌─────┐ ┌─────┐ ┌─────┐    │
%│  │  A  │ │  B  │ │  C  │ │  D  │ │  E  │    │
%│  │ ( ) │ │ ( ) │ │ ( ) │ │ ( ) │ │ ( ) │    │
%│  └─────┘ └─────┘ └─────┘ └─────┘ └─────┘    │
%│              ↑ 1 event listener             │
%│            (on parent) Efficient!           │
%└─────────────────────────────────────────────┘
%
%
%event.target vs event.currentTarget
%
%
%┌─────────────────────────────────────────────┐
%│           event.currentTarget               │ ← Parent (where handler is attached)
%│  ┌─────┐ ┌─────┐ ┌─────┐ ┌─────┐ ┌─────┐    │
%│  │  A  │ │  B  │ │  C  │ │  D  │ │  E  │    │
%│  │     │ │     │ │     │ │     │ │     │    │
%│  └─────┘ └─────┘ └─────┘ └─────┘ └─────┘    │
%│                  ↑ event.target             │ ← Actual clicked child (B)
%└─────────────────────────────────────────────┘
%
%
%Practical Example: Dynamic List
%
%html
%<ul id="shoppingList">
%<li data-id="1">Milk</li>
%<li data-id="2">Bread</li>
%<li data-id="3">Eggs</li>
%</ul>
%
%
%javascript
%// SINGLE event listener for ALL items (present and future)
%document.getElementById('shoppingList').addEventListener('click', function(event) {
%    // event.target = the actual <li> that was clicked
%    // event.currentTarget = the <ul> (this)
%
%    if (event.target.tagName === 'LI') {
%        const itemId = event.target.dataset.id;
%        console.log(`Clicked item ${itemId}: ${event.target.textContent}`);
%
%        // Remove the clicked item
%        event.target.remove();
%    }
%});
%
%// Add new items dynamically - they automatically work!
%function addItem(text) {
%    const li = document.createElement('li');
%    li.textContent = text;
%    li.dataset.id = Date.now(); // Unique ID
%    document.getElementById('shoppingList').appendChild(li);
%    // No need to add event listener - delegation handles it!
%}
%
%
%Benefits of Event Delegation:
%
%1. Memory Efficient: One listener instead of many
%2. Dynamic Content: Works with elements added later
%3. Cleaner Code: Centralized event handling
%4. Better Performance: Fewer event listeners to manage
